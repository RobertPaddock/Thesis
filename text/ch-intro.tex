%\begin{savequote}[8cm]
%\textlatin{Neque porro quisquam est qui dolorem ipsum quia dolor sit amet, consectetur, adipisci velit...}

%There is no one who loves pain itself, who seeks after it and wants to have it, simply because it is pain...
 % \qauthor{--- Cicero's \textit{de Finibus Bonorum et Malorum}}
%\end{savequote}

\chapter{\label{ch:intro}Introduction} 

\minitoc

\section{Introduction}
Nuclear fusion is the process through which small nuclei (typically isotopes of Hydrogen) fuse together to form larger nuclei, with an associated release of energy. From small acorns mighty forests grow, and this simple process is the energy source that powers the stars, formed all the elements in the universe, and will hopefully one day produce electricity for the National Grid. Yet despite decades of research on this topic, the goal of a fusion reactor remains out of reach.

There are now a large variety of approaches to fusion being studied in laboratories around the world, but these fall broadly into two main categories. Magnetic confinement fusion is what the general public knows of fusion; a tokamak uses strong magnetic fields to confine a plasma, while it is heated to extreme temperatures. If the plasma is hot enough it will begin to produce energy, and the battle is to keep hold of the plasma and stop it from expanding, touching the reactor and cooling down. The second approach, Inertial Confinement Fusion, is the focus of this thesis. In this case, the hydrogen fuel is compressed to extreme temperatures and densities (the highest ever achieved on earth). It is not possible to sustain these conditions; the pressure in the capsule quickly overcomes the compressive forces, and the capsule explodes outwards again \footnote{It is slightly confusing to describe a process where the capsule ends up exploding uncontrollably as `confinement', but this name refers to the fact that in ICF the capsule is briefly `confined' by it's inertia, as opposed to MCF where it is confined by magnetic fields}. However, in the brief period where the fuel is compressed the conditions are sufficient for fusion to occur, and the energy generated in that short time must therefore be sufficient to make the whole process worthwhile.

The compression in ICF is typically achieved using lasers. In `direct-drive' (which I focus on in this thesis) the lasers irradiate the capsule directly to provide the compression, while in `indirect-drive' (performed at the National Ignition Facility) the lasers are used to generate an x-ray bath which compresses the capsule. The end result is the same; the lasers/x-rays ablate material from the capsule, accelerating the outer layers inwards and compressing the fuel inside. This drives high temperatures and densities, ionising the fuel and turning it into a plasma.

ICF capsules can have a range of different structures and materials. The most successful tend to have a central vapour region which initially contains gaseous deuterium-tritium, surrounded by a layer of DT ice (providing the main `fuel' for the reaction), and a thin plastic shell. This thesis will focus on a new variation of this design, where the conventional ice layer is instead replaced with a low density carbon foam, saturated with liquid DT. This is equivalently known as a `liquid-layer' or a `wetted-foam layer'. Such designs have significant promise for ICF applications, in-particular through the unprecedented level of control they enable over the degree of convergence that the capsule undergoes. They also have the potential to be low-cost and simpler to produce than conventional targets, which makes them well-suited to a future fusion reactor (which requires a large supply of cheap to manufacture targets). 

\section{Structure of the thesis}

The structure of this thesis is as follows:
\begin{itemize}
    \item \hl{Chapter} will outline the fundamental theory underpinning the later chapters. This will include the basic science required to understand, describe, and simulate plasmas, and a brief overview of the basics required to understand an ICF reaction. Finally the physics of shocks in a material will be described, which is essential for the understanding of the experiment described in the latter chapters.
    \item \hl{Chapter} will describe a simulation campaign to investigate the fusion performance of ICF wetted foam capsules, restricted to a regime where instability growth is expected to be minimal. This regime and the simulation campaign itself will be described, along with the results it achieved. The nature of the simulations used will also be discussed.
    \item \hl{Chapter} will then focus on alternative techniques that could be used to build upon the results of \hl{Chapter}. This will include further simulation campaigns investigating different laser drivers, and the use of `auxilliary heating' through electron beams.
    \item The next two chapters will then focus on experimental work investigating the equation of state of a TMPTA plastic foam relevant to these previous designs. \hl{Chapter} will focus on the principles of the experiment, and the preparation work performed beforehand, while \hl{Chapter} will focus on the data analysis and interpretation of the results.
    \item Finally, \hl{Chapter} will summarise and conclude the thesis. There are also two appendices, which will focus on the operation of key code for this work, and on further experimental details.
\end{itemize}

\section{Author's contributions}

The work presented in this thesis is mostly my own, but there are places where I have relied heavily on the works of others, or worked collaboratively to achive the results quoted. I have highlighted wherever this has occurred, but also provide a summary here of these contributions.

The third-harmonic simulation campaign in \hl{Chapter}, along with the simulations of the 'hydrodynamic-equivalent' capsules and the auxilliary heating of the third-harmonic capsules, were conducted entirely by myself. I also wrote the code to support this - the meshing algorithm for the capsules, the code to produce the necessary input decks, and the analysis code that I used. However, I am grateful for the guidance and assistance of Dr Robbie Scott when it comes to these Hyades simulations - and the physics parameters I used for these simulations (such as radiation groups, ionisation models, flux limiters etc.) were recommended by him, and based largely on his own work and experience. Robbie was also invaluable for advice and troubleshooting on the Hyades and H2D simulations throughout this thesis. The benchmarking of Hyades discussed in \hl{Section} was performed by me, using data provided by Dr Alex Zylstra from LLNL. I also performed all the simulations required for the hydrodynamic equivalent capsules, and to investigate the auxilliary heating of the third harmonic capsules.

In \hl{Chapter}, some of the work was done collaboratively with students under my supervision. I performed initial two-colour simulations to investigate and prove the concept. The bulk of the simulation campaign for the ArF and two-colour simulations were then performed by Heath Martin and Rusko Ruskov (two summer students in our group), who under my supervision used my code and the procedure I developed and outlined in \hl{Chapter} to generate the quoted results. Two further summer students under my supervision, Tat Li and Eugene Kim, also used my code and followed my procedure to perform the auxiliary heating simulations for the ArF capsules (again under my supervision). Tat Li also worked with me  to perform the 'maximum areal density' optimisations. When discussing the auxiliary heating, I have also referenced subsequent results from Jordan Lee, Rusko Ruskov and Heath Martin - this was not my work, and is included for increased understanding and completeness.

The experiment was a large European collaboration, and many people contributed to it's success. I wrote the proposal (with input and suggestions from all the collaborators, but particularly Peter Norreys, John Pasley, and Dan Eakins). I performed most of the work done in the planning stage - in terms of developing the design for the optical system, working out the spatial configuration, and performing simulations to finalise target design. This was supplemented by further simulation efforts from Piotr R\k{a}czka, Paul Neumayer, and Artem Martynenko. We also held regular meetings where all the collaborators provided input and feedback. The experiment itself was conducted by a large number of the collaborators over a two week period, with the assistance of the experimental staff. Particular thanks is due to Matthew Oliver, who provided a great deal of support (even above what would be expected in his role as link scientist), and Paul Mabey, who acted as target area operator for a two week period. I was the scientific lead throughout the whole run, and also served as target area operator for the final two weeks. I performed the majority of the analysis - again aided with simulations from Piotr R\k{a}czka, Paul Neumayer, and Artem Martynenko. I have included a number of Piotr's 2D flash simulations in the experiment analysis chapter.

\section{List of peer-reviewed first author publications}

\begin{itemize}
\item Paddock, R. W., Martin, H., Ruskov, R. T., Scott, R. H. H., Garbett, W., Haines, B. M., Zylstra, A. B., Aboushelbaya, R., Mayr, M. W., Spiers, B. T., Wang, R. H. W., and Norreys, P. A. (2021). One-dimensional hydrodynamic simulations of low convergence ratio direct-drive inertial confinement fusion implosions. Philosophical Transactions of the Royal Society A: Mathematical, Physical and Engineering Sciences, 379(2189), 20200224.
\item Paddock, R. W., Martin, H., Ruskov, R. T., Scott, R. H. H., Garbett, W., Haines, B. M., Zylstra, A. B., Campbell, E. M., Collins, T. J. B., Craxton, R. S., Thomas, C. A., Goncharov, V. N., Aboushelbaya, R., Feng, Q. S., Von Der Leyen, M. W., Ouatu, I., Spiers, B. T., Timmis, R., Wang, R. H. W., and Norreys, P. A. (2022). Pathways towards break even for low convergence ratio direct-drive inertial confinement fusion. J. Plasma Phys, 88, 905880314. \textbf{(Featured article)}
\end{itemize}

\section{List of conference presentations}

\begin{itemize}
\item Poster at `Prospects for high gain inertial fusion energy' discussion meeting at the Royal Society, 02 - 03 March 2020.
\item Invited AWE plasma physics colloquium seminar (virtual), 07 January 2021.
\item Oral presentation at AWE Physics Student Conference 2021 (virtual), 14-15 April 2021.
\item Poster at European Physical Society (virtual), 21-25 June 2021.
\item Oral presentation at APS DPP 2021 (virtual), 8-12 November 2021.
\item Oral presentation at AI4ICF (in-person), Oxford, 6-7 December 2021.
\item Oral presentation at OxCHEDS (in-person), 21-22 March 2022
\item Oral presentation at AWE Materials and Analytical Student Conference 2022 (in-person), 4th-5th April 2022. 
\item Oral presentation at DDFIW 2022 (in-person), 3-5 May 2022.
\item Oral presentation at ECLIM 2022 (in-person), 19-23 September 2022.
\item Invited ALP seminar at Oxford (in-person), 10 October 2022.
\item Oral presentation at APS DPP 2022 (in-person), 17-21 October 2022.
\item Invited CFSA seminar at Warwick (in-person), 1 November 2022.
\item Poster and oral presentation at HPL 2022 (in-person), 19-21 December 2022.
\end{itemize}


