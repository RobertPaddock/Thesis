%\begin{savequote}[8cm]
%\textlatin{Neque porro quisquam est qui dolorem ipsum quia dolor sit amet, consectetur, adipisci velit...}

%There is no one who loves pain itself, who seeks after it and wants to have it, simply because it is pain...
% \qauthor{--- Cicero's \textit{de Finibus Bonorum et Malorum}}
%\end{savequote}

\chapter{Introduction} 

\minitoc


\section{Introduction}
Nuclear fusion is the process through which small nuclei (typically isotopes of Hydrogen) fuse together to form larger nuclei, with an associated release of energy. This basic process is the energy source of the stars, and the focus of a significant research effort which hopes to harness it for electricity production. Yet despite decades of effort, the goal of a fusion reactor has so far remained out of reach.

There are a variety of approaches to fusion being studied in laboratories around the world, but these fall broadly into two main categories. In magnetic confinement fusion (MCF), a reactor called a `tokamak' uses strong magnetic fields to confine a plasma as it is heated to extreme temperatures. Fusion reactions will occur in the hot plasma, and if it can be confined for long enough timescales these will occur in such numbers that the fusion energy produced is larger than the energy required to power the reactor, resulting in net energy production. The second approach, inertial confinement fusion (ICF) is the focus of this thesis. In ICF, a capsule containing hydrodgen fuel is compressed to extreme temperatures and densities - the highest ever achieved on earth. Such conditions cannot be sustained; as the capsule implodes the pressure inside quickly rises, and this will eventually overcome the drive pressure and cause the capsule to explode outwards (the fuel is `confined' only briefly, by it's own inertia). However, if the temperature and density are high enough in the brief period where the fuel is maximally compressed, large numbers of fusion reactions can be generated and the energy produced by these reactions can be larger than that required to drive the implosion. 

The compression in ICF is typically achieved using lasers. In `direct-drive' (the main focus of this thesis) the lasers irradiate the capsule directly, while in `indirect-drive' the lasers are incident instead on the inside of a hohlraum, producing a bath of x-rays which drive the capsule compression. The end result is the same; the lasers/x-rays ablate material from the capsule, accelerating the rest of the capsule inwards and compressing the fuel inside. This drives high temperatures and densities, ionising the fuel and turning it into a plasma.

The highest-energy facility performing ICF implosions is the National Ignition Facility (NIF), at Lawrence Livermore National Laboratory in California. At the NIF, 192 beams provide up to 2.1 MJ of laser energy in a polar beam configuration designed for indirect-drive experiments. The NIF has now been in operation for just over a decade, during which it has demonstrated a continual improvement in achieved fusion performance. This recently resulted in a shot in December 2022 where it achieved break-even (where the fusion energy output was larger than the laser energy input) for the first time. This is a momentous achievement, and represents the first time net energy production has been achieved in any fusion experiment (MCF, ICF, or otherwise). This has sparked fresh excitement about ICF, and makes it an exciting time to be working in the field.

Despite this, there remains significant work to be done on this topic. Firstly, for a successful inertial fusion energy (IFE) power plant, a much higher fusion output (a gain, or energy out vs energy in, of 50-100) is required \cite{Campbell2017}; this is necessary to cover the energy use in generating the laser beams and inside the power plant, and to make the whole operation financially viable. The cost of such a facility would be large, and a range of analyses have been performed to estimate the cost per unit energy required to make such power stations competitive \cite{Tynan2020, Gi2020}. The reactor would also need to operate at high repetition rate, on the order of multiple shots a second, as opposed to the roughly one shot a day that the NIF can currently provide. Along with improvements to the laser technology, and the design of a reactor that could deliver targets at this rate and deal with the high neutron flux, this would require a supply of targets that can be produced quickly and to the required levels of precision at low cost \cite{Nuckolls2010}.

Conventional ICF targets (including those used in the recent ignition shot) tend to use a frozen layer of DT as the main bulk of the fuel. This DT-ice surrounds a central region containing DT vapour, and is in turn surrounded by a shell of ablator material. Such designs enabled impressive performance, but each target is expensive and time-consuming to produce \cite{Goncharov2020}. `Wetted-foam' or `liquid-layer' targets are being explored as a potential alternative. Rather than use an ice-layer, these targets contain a low-density carbon foam saturated with liquid DT, where the liquid is held in the foam by capillary forces. Such targets could solve some of the manufacturing challenges associated with ice-layers, and can be 3D printed with the potential for rapid mass-production \cite{Olson2021}.

The interest in wetted-foam capsules is not solely based on these practical reasons, as they also have a range of potential advantages for fusion performance. Of particular interest for this thesis is the opportunity they provide to control capsule convergence. Ice layer targets must be fielded at temperatures lower than 19 K, in order to ensure that the DT remains frozen. Wetted-foam capsules however are fielded at higher temperatures between 20 - 26 K (corresponding to the range of temperatures over which DT is in liquid form) \cite{Olson2016}. Increasing this temperature leads to a higher density of the DT vapour in the central void of the capsule (between 0.6 and 4.0 \unit{\milli\gram\per\centi\meter\cubed}), and this higher density leads to reduced compression and convergence. By varying the initial temperature of wetted-foam capsules, the amount of convergence experienced in the implosion can be controlled and explored; this is of interest, since low-convergence implosions are known to be more robust to hydrodynamic instability growth.

%In addition to these practical benefits, wetted-foam capsules also allow interesting new avenues of research to be investigated. The use of liquid rather than ice DT means that the capsules are fielded at higher temperatures relative to capsules containing DT ice (19 - 25 K, rather than below the 19K DT freezing point for ice capsules). This results in higher vapour densities and pressures in the central DT vapour, which can be used to generate low convergence ratio implosions - i.e., implosions when the capsule compression is reduced compared to a normal ICF implosion. This leads to lower ideal fusion yields, but such implosions have been shown to be less susceptible to the hydrodynamic instability growth that plagues ICF implosion.

%This thesis covers a range of work of relevance to wetted-foam capsules and their potential for generating low-convergence ratio implosions. The first half of this thesis covers simulation efforts to explore the potential fusion performance of wetted foam capsules in an expected low-instability regime (of which low-convergence ratio is a key part), using a variety of different techniques. In the second half, an experiment is planned, conducted and analysed, which explores the compression behaviour of a foam material relevant to these earlier designs, to improve the modelling of foam materials and in preparation for a potential future experiment based on the results in the first half.

%covers two major pieces of work, based around wetted foam capsules and their potential for low-convergence ratio implosions. First, a range of simulations are used to investigate the possible fusion performance of such capsules (using a variety of different techniques and technologies) in a low-instability regime defined by low convergence ratio. In the second, an experiment is planned, performed and analysed to investigate the compression behaviour of a foam material relevant to these designs, to inform future modelling and simulation efforts of such foams (in preparation for a future experiment based on approaches suggested in the first chapter).

\section{Structure of the thesis}

This thesis investigates the potential performance of direct-drive ICF implosions of wetted-foam capsules, along with studying the compression behaviour of foams relevant to these designs. It is split into the following chapters:

%\begin{itemize}
	%\item Chapter \ref{ch-theory} will describe some of the key theory used in the later chapters (covering basic plasma physics, mathematical descriptions of plasmas and their implementation in simulation codes, an overview of ICF implosions and instability growth, and a description of shock physics).
	%\item Chapter \ref{ch:definitions} will define and discuss some of the key terms and implosion parameters used throughout this thesis, and implemented in the code used to analyse the simulations.
	%\item Chapter \ref{ch-lowCR} will discuss the role of low convergence ratio in minimising instability growth, and a simulation campaign in an expected low-instability regime investigating the possible performance of wetted foam capsules.
	%\item Chapter \ref{ch-FurtherSims} will present further simulations extending the results of the previous chapters. This will include the impact of the equation of state model used, designs for surrogate targets to enable room temperature experiments based on these designs, and then further simulation campaigns investigating the performance that can be obtained when alternative laser drivers or auxiliary heating techniques are applied within this regime.
	%\item The next two chapters will then focus on experimental work investigating the equation of state of a TMPTA plastic foam relevant to these previous designs. Chapter \ref{ch-experiment} will focus on the principles of the experiment, and the preparation work performed beforehand, while Chapter \ref{ch-experimentAnalysis} will focus on the data analysis and interpretation of the results.
	%\item Chapter \ref{ch:Conclusion} will then summarise and conclude the thesis. There are also three appendices, which will focus on the operation of key code for this work, some bench-marking of the simulations, and further details of the experimental setup.
%\end{itemize}

\begin{itemize}
	\item Chapter \ref{ch-theory} will describe the key theory of plasmas, ICF and shock propagation through materials.
	\item Chapter \ref{ch:definitions} will define a number of key terms and implosion parameters used throughout this thesis, and discuss how exactly they are evaluated for the simulations presented in later chapters.
	\item Chapter \ref{ch-lowCR} will discuss the role of low convergence ratio in minimising instability growth, and present a simulation campaign investigating the performance of wetted-foam capsules in an expected low-instability regime.
	\item Chapter \ref{ch-FurtherSims} will present further simulations extending the results of the previous chapter. It will investigate the impact of the equation of state model used, designs for surrogate targets to enable room temperature experiments based on these designs, and then include further simulation campaigns investigating the performance that can be obtained when alternative laser drivers or auxiliary heating techniques are applied within the low-instability regime.
	\item The next two chapters will focus on experimental work investigating the equation of state of a TMPTA plastic foam relevant to these previous designs. Chapter \ref{ch-experiment} will describe the principles, diagnostics, and preparation work performed for the experiment, while Chapter \ref{ch-experimentAnalysis} will focus on the data analysis and interpretation of the results.
	\item Chapter \ref{ch:Conclusion} will then summarise the thesis. There are also three appendices, which will focus on the operation of key code for this work, bench-marking of the simulations, and further details of the experimental setup.
\end{itemize}

\section{Author's contributions}

The work in this thesis is largely my own, but there are places where I have included the work of collaborators, or the results of students who I supervised. I have indicated this wherever it has occurred, but also provide a brief summary here of these contributions.

Throughout the thesis, I relied upon the guidance and assistance of Dr Robbie Scott when running simulations in \texttt{HYADES}/\texttt{h2D}. In particular, the physics models used (such as for ionisation, radiation transport etc.) were based on his recommendations. The benchmarking of \texttt{HYADES} presented in Appendix \ref{app:benchmark} was performed using data provided by Dr Alex Zylstra and Dr Brian Haines, who also provided advice during this process.

While performing this work I supervised a number of summer and master's students. These students helped to perform a number of the simulation campaigns discussed in this thesis, using the code and methodology I had created for this purpose and under my instruction; this allowed us to explore a wider range of implosions than would otherwise have been possible. This included Heath Martin and Rusko Ruskov (who contributed to the simulations of alternative laser drivers), along with Tat Li and Eugene Kim (who applied auxilliary heating to some of these alternative laser driver results, and in Tat's case also optimised two implosions for maximum areal density).

At the end of the section on auxiliary heating, I have provided a brief summary of work performed on this topic by Jordan Lee, Heath Martin and Rusko Ruskov. This was not my work, and was performed after the results presented in that chapter were generated; but their results have changed our understanding of how the scheme works, and so they are discussed briefly to give an up-to-date and accurate representation of the heating process.

%I supervised a number of summer and master's students during my DPhil, who under my instruction helped to extend my results to a wider range of implosions. After I had performed optimisation of the third-harmonic implosions, Heath Martin and Rusko Ruskov repeated the optimisation to explore ArF and two-colour capsules, using code that I had adapted for this purpose. Tat Li and Eugene Kim later applied auxiliary heating to a selection of these ArF and two-colour implosions (following work I had personally done to apply such heating to the third-harmonic implosions). Tat Li also optimised two capsules to produce maximal areal density rather than yield (again, using code I had produced for this purpose). Their contributions meant that we could explore a much larger range of implosions than would otherwise have been possible. In Chapter \ref{ch-FurtherSims}, I have also included a section describing subsequent work performed by Jordan Lee, Heath Martin and Rusko Ruskov into how exactly the auxiliary heating process works; this is not my work, and was included only to give a more complete and accurate understanding of the heating mechanism.

%The third-harmonic simulation campaign in \hl{Chapter}, along with the simulations of the 'hydrodynamic-equivalent' capsules and the auxilliary heating of the third-harmonic capsules, were conducted entirely by myself. I also wrote the code to support this - the meshing algorithm for the capsules, the code to produce the necessary input decks, and the analysis code that I used. However, I am grateful for the guidance and assistance of Dr Robbie Scott when it comes to these Hyades simulations - and the physics parameters I used for these simulations (such as radiation groups, ionisation models, flux limiters etc.) were recommended by him, and based largely on his own work and experience. Robbie was also invaluable for advice and troubleshooting on the Hyades and H2D simulations throughout this thesis. The benchmarking of Hyades discussed in \hl{Section} was performed by me, using data provided by Dr Alex Zylstra from LLNL. I also performed all the simulations required for the hydrodynamic equivalent capsules, and to investigate the auxilliary heating of the third harmonic capsules.

%In \hl{Chapter}, some of the work was done collaboratively with students under my supervision. I performed initial two-colour simulations to investigate and prove the concept. The bulk of the simulation campaign for the ArF and two-colour simulations were then performed by Heath Martin and Rusko Ruskov (two summer students in our group), who under my supervision used my code and the procedure I developed and outlined in \hl{Chapter} to generate the quoted results. Two further summer students under my supervision, Tat Li and Eugene Kim, also used my code and followed my procedure to perform the auxiliary heating simulations for the ArF capsules (again under my supervision). Tat Li also worked with me  to perform the 'maximum areal density' optimisations. When discussing the auxiliary heating, I have also referenced subsequent results from Jordan Lee, Rusko Ruskov and Heath Martin - this was not my work, and is included for increased understanding and completeness.

The experiment was a large collaboration, and many groups and individuals contributed to it's development, operation and analysis. I performed the work presented unless stated otherwise, but was supported throughout by input from our collaborators during the proposal writing, through regular planning meetings in the year before the experiment, and in analysis meetings afterwards. A large number of collaborators attended the experiment itself (including Matthew Oliver, who helped out a great deal in his role as link scientist, and Paul Mabey, who acted as target area operator for two weeks). I have also included a few results generated by these collaborators in the thesis. The VISAR was produced by Professor Daniel Eakins, and he provided the analysis code which was used to calculate shock velocity from VISAR fringe motion. Michael Woodward produced all the CAD models presented throughout the experiment chapters. Rati Goshadze, Dr Valentin  Karasiev, and Professor Suxing Hu performed the DFT simulations calculating the foam reflectivity at a range of pressures, which allowed me to calculate the grey-body foam shock temperatures. Finally, Professor Piotr R\k{a}czka, Dr Paul Neumayer, and Dr Artem Martynenko performed pre- and post-experiment simulations to support my own simulations; the results of some of these are included in \ref{fig:SimulationPlot}, and some of Piotr's 2D \texttt{FLASH} simulations are also presented in Figures \ref{fig:SimSubPlot} and \ref{fig:GapSims}

%The experiment was a large European collaboration, and many people contributed to it's success. I wrote the proposal (with input and suggestions from all the collaborators, but particularly Peter Norreys, John Pasley, and Dan Eakins). I performed most of the work done in the planning stage, including developing the design for the optical system, working out the spatial configuration, and performing simulations to finalise target design. This was supplemented by further simulation efforts from Piotr R\k{a}czka, Paul Neumayer, and Artem Martynenko. We also held regular meetings where all the collaborators provided input and feedback. The experiment itself was conducted by a large number of the collaborators over a two week period, with the assistance of experimental staff at the CLF. Particular thanks is due to Matthew Oliver, who provided a great deal of support (even above what would be expected in his role as link scientist), and Paul Mabey, who acted as target area operator for a two week period. I was the scientific lead throughout the whole run, and also served as target area operator for the final two weeks. I performed the data analysis, again aided with simulations from Piotr R\k{a}czka, Paul Neumayer, and Artem Martynenko. I have included some of these simulations in this thesis (including Piotr's 2D flash simulations). Rati Goshadze, Valentin  Karasiev, and Suxing Hu also very kindly performed the DFT simulations for the foam reflectivity. The VISAR was designed and built by Professor Dan Eakins, who also provided the VISAR analysis code which was used to analyse the fringe motion.

\section{List of peer-reviewed first author publications}

\begin{itemize}
	\item Paddock, R. W., Martin, H., Ruskov, R. T., Scott, R. H. H., Garbett, W., Haines, B. M., Zylstra, A. B., Aboushelbaya, R., Mayr, M. W., Spiers, B. T., Wang, R. H. W., and Norreys, P. A. (2021). One-dimensional hydrodynamic simulations of low convergence ratio direct-drive inertial confinement fusion implosions. Philosophical Transactions of the Royal Society A: Mathematical, Physical and Engineering Sciences, 379, 20200224.
	\item Paddock, R. W., Martin, H., Ruskov, R. T., Scott, R. H. H., Garbett, W., Haines, B. M., Zylstra, A. B., Campbell, E. M., Collins, T. J. B., Craxton, R. S., Thomas, C. A., Goncharov, V. N., Aboushelbaya, R., Feng, Q. S., Von Der Leyen, M. W., Ouatu, I., Spiers, B. T., Timmis, R., Wang, R. H. W., and Norreys, P. A. (2022). Pathways towards break even for low convergence ratio direct-drive inertial confinement fusion. Journal of Plasma Physics, 88, 905880314. \textbf{(Featured article)}
	\item Paddock, R. W., von der Leyen, M. W., Aboushelbaya, R., Norreys, P. A., Chapman, D. J., Eakins, D. E., Oliver, M., Clarke, R. J., Notley, M., Baird, C. D., Booth, N., Spindloe, C., Haddock, D., Irving, S., Scott, R. H. H., Pasley, J., Cipriani, M., Consoli, F., Albertazzi, B., … Hu, S. X. (2023). Measuring the principal Hugoniot of inertial-confinement-fusion-relevant TMPTA plastic foams. Physical Review E, 107, 025206.
\end{itemize}

\section{List of conference presentations}

\begin{itemize}
	\item Poster at `Prospects for high gain inertial fusion energy' discussion meeting at the Royal Society, 02 - 03 March 2020.
	\item Invited AWE plasma physics colloquium seminar (virtual), 07 January 2021.
	\item Oral presentation at AWE Physics Student Conference 2021 (virtual), 14-15 April 2021.
	\item Poster at European Physical Society (virtual), 21-25 June 2021.
	\item Oral presentation at APS DPP 2021 (virtual), 8-12 November 2021.
	\item Oral presentation at AI4ICF (in-person), Oxford, 6-7 December 2021.
	\item Oral presentation at OxCHEDS (in-person), 21-22 March 2022
	\item Oral presentation at AWE Materials and Analytical Student Conference 2022 (in-person), 4th-5th April 2022. 
	\item Oral presentation at DDFIW 2022 (in-person), 3-5 May 2022.
	\item Oral presentation at ECLIM 2022 (in-person), 19-23 September 2022.
	\item Invited ALP seminar at Oxford (in-person), 10 October 2022.
	\item Oral presentation at APS DPP 2022 (in-person), 17-21 October 2022.
	\item Invited CFSA seminar at Warwick (in-person), 1 November 2022.
	\item Poster and oral presentation at HPL 2022 (in-person), 19-21 December 2022.
	\item Poster at NIF user group 2023 (in-person), 21-23 February 2023.
\end{itemize}


