Recent experiments using DT-wetted-foams in inertial confinement fusion capsules have demonstrated that the convergence of the implosion can be controlled through varying the temperature that the capsule is fielded at. They have also demonstrated low amounts of hydrodynamic instability growth at low convergence ratio. In this thesis, a range of research relevant to this is presented. Simulation campaigns are discussed which explore the fusion gains that could be achieved in experiments involving such capsules, in a newly-identified regime (based around limits to convergence ratio and other implosion parameters) where instability growth is expected to be minimal. Performance is explored first for third-harmonic of Nd:glass laser drivers, and then higher-frequency ArF drivers and a novel `two-colour' laser scheme. Further simulations are then presented which explore how such capsules could be heated using electron beams, and the impact this has on fusion performance is discussed. `Hydrodynamic-equivalent' capsules are also presented, which are surrogate targets where the DT-wetted foam layer is replaced instead with an equivalent-density layer of dry foam. Such capsules have comparable hydrodynamic performance, and could facilitate room-temperature experiments to validate the low-instability regime.

The equation of state of these foams are not well characterised. An experiment is therefore presented which measured principal Hugoniot data of TMPTA foam at 260 \unit{\milli\gram\per\centi\meter\cubed}, relevant to the foam layer in the hydrodynamic-equivalent capsule. VISAR was used to measure shock transit times through an alpha-quartz reference layer and a TMPTA foam layer in a multi-layer target and impedance matching performed to enable calculation of the foam shock state, while the foam shock temperature was estimated using streaked optical pyrometry. The results demonstrate that for the pressure range probed in the experiment the TMPTA foam was well described by existing QEOS and SESAME models, suggesting that this foam is well-approximated as a low-density homogenous plastic.