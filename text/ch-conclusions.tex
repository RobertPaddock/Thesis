%\begin{savequote}[8cm]
%\textlatin{Neque porro quisquam est qui dolorem ipsum quia dolor sit amet, consectetur, adipisci velit...}

%There is no one who loves pain itself, who seeks after it and wants to have it, simply because it is pain...
% \qauthor{--- Cicero's \textit{de Finibus Bonorum et Malorum}}
%\end{savequote}

\chapter{Summary and future work} \label{ch:Conclusion}

\minitoc

\section{Summary}

This thesis presents a number of new results relevant to low-convergence ratio ICF implosions and the wetted-foam capsules that facilitate them. Firstly, an interesting new `low-instability' regime was identified, defined through restricting convergence ratio and other key implosion parameters, where instability growth is expected to be minimal. This regime was explored in a large simulations campaign, and promising performance was identified within it. This also established the necessary code and methods to allow other parameter spaces and implosion types to be explored in this way.

This simulation work was then expanded in a variety of different directions. The simulation campaigns were repeated to explore the fusion performance of alternative laser drivers, including higher frequency and novel `two-colour' implosions. It was demonstrated that moving to higher frequencies would allow higher gains to be generated at low energy, and would allow gains closer to that required for an IFE reactor to be achieved even within the `low-instability' regime. Further simulations explored the role that auxiliary heating of implosions using electron beams could potentially play in improving fusion performance. It was found that this technique could significantly improve the yield of sub-ignition implosions, although the potential of such a scheme is heavily dependent upon how efficiently such beams of electrons can be generated and deposited into the plasma.

In order to facilitate future experimental verification of these results and the low-instability nature of the regime, surrogate `hydrodynamic equivalent' capsules were produced. These demonstrate similar hydrodynamic behaviour to the main implosions, but can be performed at room temperature; permitting easier experiments to allow the regime to experimentally validated. Such capsules replace the wetted-foam layer with a `dry' (without DT-wetting) foam of equivalent overall density (i.e. a higher foam density, to compensate for the lack of DT).

Foams in general are poorly described by existing equation of state models, and this presents a limit on the accuracy of simulations of these `hydrodynamic equivalent' capsules, and wetted-foam capsules in general. To help address this, an experiment was performed at VULCAN to measure principal Hugoiniot data of TMPTA foam at 260 \unit{\milli\gram\per\centi\meter\cubed}. VULCAN was used to drive a shock wave through a multi-layer target containing alpha quartz (a reference material) and the TMPTA foam, and shock breakout times in the quartz and foam were measured using VISAR, while SOP was used to measure shock temperatures. The results suggested that in the pressure range of 20 - 120 \unit{\giga\pascal} probed in the experiment, the foam could be reasonably well described by equation of state models for low-density homogeneous plastic. This is encouraging, and suggests that simulations of the hydrodynamic-equivalent targets should be reasonably accurate, and this data also leads into understanding of foams generally.

\section{Future work}
There are a few obvious research directions suggested by this work. The simulation results in the first chapter suggest promising performance in a regime where instability is expected to be low, and it would be desirable to test these assumptions experimentally. The hydrodynamic-equivalent capsules were proposed for this purpose, and an experiment should be proposed based on these capsules to test the low-instability regime. This will require further simulation work in higher-dimensions, and this is being performed through collaboration with the University of Rochester. Such an experiment would provide experimental verification of this work.

There are other routes that could also be taken forward. The auxiliary heating results discussed in the Further simulations chapter shows significant promise, but much more work is required. Firstly, further simulation should be performed where PIC/VLASOV codes are used to simulate the actual heating mechanism, the results of which are then fed into a fluid code. Work is underway in our group to develop this. Further research is required into electron beams to better understand the efficiency with which these can be created and delivered to the hotspot. Ultimately, it would also be interesting to conduct an experiment where such beams are used to heat a plasma so that this process can be demonstrated experimentally.

Other follow-on experiments could also be conducted following the TMPTA shock-compression experiment. While this was an investigation of a dry-foam material, there is a real open research question regarding the equation of state of DT-wetted foam. This is a far more challenging experiment to conduct (particularly because it would need to be performed at cryogenic temperatures), but there is real interest in this topic. Such an experiment should therefore be performed as a high priority.

Finally, the simulation platform developed and used to explore the low-instability regime demonstrated a lot of potential to effectively explore fusion performance in different parameter spaces. It could easily be applied to a range of other types of implosion (as demonstrated with the high frequency and two-colour campaigns), but it is currently labour-intensive. This type of optimisation procedure is well-suited to machine learning applications, however, and this would remove the need for a user to run the simulations and significantly speed up how quickly results could be obtained. It would therefore be valuable to implement machine learning to run this optimisation, and this is again something that is currently being investigated.